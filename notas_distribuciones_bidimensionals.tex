\documentclass[11 pts, letterpaper, twosided]{article}

%%
%% Preambulo construido para documentos latex
%% Actualizado: 30.marzo.2015
%%
\usepackage[spanish,es-tabla]{babel}
        \spanishdecimal{.}
\renewcommand{\listtablename}{Índice de tablas}
\renewcommand{\tablename}{Tabla}
%%
%% ---------------- Indice y estilos de indice
\usepackage{makeidx}
        \makeindex
%
%% ---------------- Fuentes
\usepackage{bookman}
% Font Settings                                                              
%\usepackage{avant} % Use the Avantgarde font for headings
%\usepackage{times} % Use the Times font for headings
%\usepackage{mathptmx} % Use the Adobe Times Roman as the default text font together with math symbols from the Sym-bol, Chancery and Com-puter Modern fonts
\usepackage{microtype} % Slightly tweak font spacing for aesthetics         
\usepackage[utf8]{inputenc}
\usepackage[T1]{fontenc} % Use 8-bit encoding that has 256 glyphs
%%
%% ---------------- Estilos para el formato de página
%%
\usepackage[marginparwidth=3cm, marginparsep=0.3cm]{geometry}
        \geometry{%
                  tmargin=1.5in,
                  bmargin=1.5in,
                  lmargin=2in,
                  rmargin=1in}
        \reversemarginpar
%%\usepackage{float}
\usepackage{chngpage}
\usepackage{fancyhdr}
        \pagestyle{fancy}
                \fancyhf{}
                \rhead{\sc{Est. Aplicada al Diseño de Exp.}}
                \lhead{\sc\nouppercase{\leftmark}}
                \cfoot{\thepage}
                \rfoot{{\em luciano.aguilera@itcm.edu.mx\/}}
%%
%% ----------------- Figuras
%%
\usepackage[pdftex]{graphicx}
        \graphicspath{ {CAPITULOS/IMAGENES-BIOENERGIA/} }
\usepackage{eso-pic} % Se requiere para poner imágenes de fondo
\usepackage[margin=0.5in, hang, bf, small]{caption}
\usepackage{subcaption}
\usepackage{xcolor}
\usepackage{colortbl}
\definecolor{LightCyan}{rgb}{0.88,1,1}
%\usepackage{color}
\usepackage{wrapfig}
\usepackage[section]{placeins}
\usepackage{tikz}
\usepackage{rotating}
\usepackage{pdflscape}
\usepackage{afterpage}
\setlength{\fboxsep}{10pt}
\usepackage{picinpar}
%\usepackage{palatino,avant}


%%
%% ---------------- Estilos para el referenciado
\usepackage{cite}
\usepackage[sort]{natbib}
\usepackage{apalike}
\usepackage{bibentry}
\usepackage{url}
\usepackage{marginnote}
\renewcommand*{\marginfont}{\color{red}\sffamily\footnotesize}

%% ---------------- Estilos de referenciado utilizando Biblatex
%\usepackage[backend=bibtex, natbib=true, style=authoryear]{biblatex}
%\usepackage[backend=bibtex, natbib=true, style=apa]{biblatex}
%\usepackage[backend=biber, natbib=true, sortlocale=auto, style=apa]{biblatex}
%\usepackage{biblatex}
%     \bibliography{/home/luciano/Documentos/BIBLIOTECA/cursos.bib}
%     \addbibresource{cursos.bib}
%
%% ----------------- Problemas y ejercicios
% \usepackage[thmmarks,standard,thref]{ntheorem}
%     \theoremseparator{.}
%     \theorembodyfont{\upshape}
%     \newtheorem{prob}{Problema}[chapter]

\usepackage{answers}

%% ----------------- Estilos Matemáticos y simbolos
\usepackage{amssymb}
\usepackage{amsthm}
     \theoremstyle{definition}
     \newtheorem{ejmp}{Ejemplo}[section]
\usepackage{amsmath}
\usepackage{amsbsy}
\usepackage{amsfonts}
\usepackage{eurosym}
\usepackage[version=4]{mhchem} % Simbología de reacciones químicas
%%
%%
%% ------------------Estilos de tablas
\usepackage{xtab}
%\usepackage{longtable}
%\usepackage{supertabular}
\usepackage{booktabs} % Mejora la apariencia de las líneas en el ambiente "tabular"
\usepackage{multicol}
\usepackage{multirow}
\usepackage{array}
\newcolumntype{L}[1]{>{\raggedright\let\newline\\\arraybackslash\hspace{0pt}}m{#1}}
\newcolumntype{C}[1]{>{\centering\let\newline\\\arraybackslash\hspace{0pt}}m{#1}}
\newcolumntype{R}[1]{>{\raggedleft\let\newline\\\arraybackslash\hspace{0pt}}m{#1}}
%%
%% ------------------Estilos de párrafos
\usepackage{parskip}
\parindent0pt 
\renewcommand{\baselinestretch}{1.3}
% Control de líneas huerfanas y líneas viudas
\clubpenalty=10000
\widowpenalty=10000
%\setlength{\parskip}{\baselinestretch}
\setlength{\parskip}{1.3\baselineskip}
\usepackage{ragged2e}
\usepackage{verbatim}
\usepackage{textcase}
\usepackage{exercise}

%% ----------------- Listas

\usepackage[inline]{enumitem} % Personaliza listas
                \setlist{nolistsep} % Reduce el espaciado en las listas

%%
%% ----------------- Comandos que requiere Fancyheader
%%                   antes de procesar el texto
%%
\renewcommand{\chaptermark}[1]{%
        \markboth{\thechapter.\ #1}{}}
%
\everymath{\displaystyle}
%%
%% Definición de colores y ambientes
\definecolor{color1}{RGB}{44,42,37}       % Borde general
\definecolor{color2}{RGB}{240,240,240}    % Fondo del cuerpo
\definecolor{color3}{RGB}{44,42,37}       % Fondo del Encabezado
\definecolor{color4}{RGB}{240,240,240}    % Borde del Encabezado


%%
%%
\hyphenation{problema presente moléculas especialmente plantas}



\title{Distribuciones bidimensionales\\
      {\normalsize \textsc{Estadística Aplicada al Diseño de Experimentos}}}
  \author{Dr. Luciano Aguilera Vázquez}

\begin{document}

\maketitle

\section{Introducción}

Cuando existen dos características de un mismo elemento de la
población (altura y peso, dos asignaturas, logitud y latitud), de
forma general, si se estudian sobre una misma población y se miden por
las mismas unidades estadísticas una variable $X$ y una variable $Y$,
se obtienen series estadísticas de las variabels $X$ e
$Y$. Considerando simultáneamente las dos series, se suele decir que
estamos ante una variable estadística bidimensional.

\section[Dist. est. bidimensionales]{Distribuciones estadísticas bidimensionales}

\subsection{Tablas de doble entrada o de contingencia}

Sea una poblacion estudiada simultáneamente según dos características
$X$ e $Y$; que representaremos genéricamente como $(x_i;\quad
y_j;\quad n_{ij})$, donde $x_i$, $y_j$ son dos valores cualesquiera y
$n_{ij}$ es la frecuencia absoluta conjunta del valor $i$-ésimo de $X$
con el $j$-ésimo de $Y$.

La distribucion de frecuencias bidimensional de ($X$, $Y$), es el
conjunto de valores

${(x_i, y_j); n_{ij}} i = 1, ..., p; j = 1, ..., q$

tal que

$\sum_i^p\sum_j^q n_{ij}= N$ o equivalente $\sum_i^p\sum_j^q f_{ij} = 1$

Una forma de disponer estos resultados es la conocida como tabla de
doble entrada o tabla de contingencia, la cual podemos representar
como sigue:

\begin{center}
    \begin{tabular}[c]{|c||c|c|c|c|c|c||c|}
      \hline
      \qquad &  $y_1$ & $y_2$ & $\dots$ & $y_j$ & $\dots$ & $y_k$ & $n_i$\\
      \hline
      \hline
      $x_1$  &  $n_{11}$ & $n_{12}$ & $\dots$ & $n_{1j}$ & $\dots$ & $n_{1k}$ & $n_{1\cdot}$\\
      \hline
      $x_2$  &  $n_{21}$ & $n_{22}$ & $\dots$ & $n_{2j}$ & $\dots$ & $n_{2k}$ & $n_{2\cdot}$\\
      \hline
      $x_3$  &  $n_{31}$ & $n_{32}$ & $\dots$ & $n_{3j}$ & $\dots$ & $n_{3k}$ & $n_{3\cdot}$\\
      \hline
      $\vdots$  &  $\vdots$ & $\vdots$ & $\dots$ & $\vdots$ & $\dots$ & $\vdots$ & $\dots$\\
      \hline
      $x_i$  &  $n_{i1}$ & $n_{i2}$ & $\dots$ & $n_{ij}$ & $\dots$ & $n_{ik}$ & $n_{i\cdot}$\\
      \hline
      $\vdots$  &  $\vdots$ & $\vdots$ & $\dots$ & $\vdots$ & $\dots$ & $\vdots$ & $\dots$\\
      \hline
      $x_h$  &  $n_{h1}$ & $n_{h2}$ & $\dots$ & $n_{hj}$ & $\dots$ & $n_{hk}$ & $n_{h\cdot}$\\
      \hline
      \hline
      $x_{\cdot j}$  &  $n_{\cdot 1}$ & $n_{\cdot 2}$ & $\dots$ & $n_{\cdot j}$ & $\dots$ & $n_{\cdot k}$ & $N$\\
      \hline
    \end{tabular}
\end{center}


Por ejemplo, cuando se trate de la frecuencia $n_{ij}$ nos indica el
número de veces que aparece $x_1$ conjuntamente con $y_1$; el valor de
frecuencia $n_{12}$, indica la frecuencia conjunta de $x_1$ con $y_2$,
y así se continua con la interpretación de estos valores. $N$
representa el total de elementos de la población y $n_{\cdot j}$
representa las frecuencias marginales en la fila, es decir de $y$ y
$n_{i\cdot}$ las de $x$.

Uno de los objetivos del análisis de distribuciones bidimensionales es
estudiar si existe una asociación o relación entre las variables $X$ y
$Y$. A partir de este tipo de distribución, también es posible obtener
distribuciones unidimensionales de dos tipos: \emph{marginales} y \emph{condicionadas}.

\section[Representaciones gráficas]{Representaciones gráficas:
    diagrama de dispersión o nube de puntos}

Este tipo de frecuencias se representan en ejes coordenados, una de
las variables en el eje $X$ y la otra en el eje $Y$, y para indicar el
número de coincidencias, se coloca el número de frecuencias entre
paréntesis en las coordenadas, o bien símbolos diferentes. En algunos
casos se recurre a círculos de diferentes tamaños de acuerdo a la
frecuencia, ubicando el centro de los mismos en el punto de
coordenadas.

\begin{center}
    \includegraphics[scale=1.8]{nube_puntos.png}\\
    Los número indican la frecuencia por par coordenado
\end{center}

\begin{center}
    \includegraphics[scale=1.8]{nube_puntos_02.png}\\
Los diferentes colores y tamaños de los círculos permiten
    hacer inferencias sobre la frecuencia de cada coordenada
\end{center}

\begin{center}
    \includegraphics[scale=2.8]{nube_puntos_03.png}\\
    En este caso, el grado de transparencia permite hacer inferencias
    sobre la frecuencia de cada par ordenado de que se trate.
\end{center}

\section{Distribuciones marginales}

Apartir de una distribución bidimensional se pueden obtener 2
distribuciones unidimensionales MARGINALES: marginal de $X$ y marginal
de $Y$. 

En el caso de $X$ se tiene

\begin{center}
    \begin{tabular}[c]{|c||c|c|c|c|c|c||c|}
      \hline
      \qquad &  $y_1$ & $y_2$          & $n_i$\\
      \hline
      \hline
      $x_1$  &  $n_{11}$ & $n_{12}$    & $n_{1\cdot}$\\
      \hline
      $x_2$  &  $n_{21}$ & $n_{22}$    & $n_{2\cdot}$\\
      \hline
      $x_3$  &  $n_{31}$ & $n_{32}$    & $n_{3\cdot}$\\
      \hline
      $\vdots$  &  $\vdots$ & $\vdots$ & $\dots$\\
      \hline
      $x_i$  &  $n_{i1}$ & $n_{i2}$    & $n_{i\cdot}$\\
      \hline
      $\vdots$  &  $\vdots$ & $\vdots$ & $\dots$\\
      \hline
      $x_h$  &  $n_{h1}$ & $n_{h2}$    & $n_{h\cdot}$\\
      \hline
      \hline
      $x_{\cdot j}$  &  $n_{\cdot 1}$  & $N$\\
      \hline
    \end{tabular}
\end{center}

\end{document}

%%% Local Variables:
%%% mode: latex
%%% TeX-master: t
%%% End:
