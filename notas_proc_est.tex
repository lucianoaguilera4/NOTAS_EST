\documentclass[11 pts, letterpaper, twosided]{article}

%%
%% Preambulo construido para documentos latex
%% Actualizado: 30.marzo.2015
%%
\usepackage[spanish,es-tabla]{babel}
        \spanishdecimal{.}
\renewcommand{\listtablename}{Índice de tablas}
\renewcommand{\tablename}{Tabla}
%%
%% ---------------- Indice y estilos de indice
\usepackage{makeidx}
        \makeindex
%
%% ---------------- Fuentes
\usepackage{bookman}
% Font Settings                                                              
%\usepackage{avant} % Use the Avantgarde font for headings
%\usepackage{times} % Use the Times font for headings
%\usepackage{mathptmx} % Use the Adobe Times Roman as the default text font together with math symbols from the Sym-bol, Chancery and Com-puter Modern fonts
\usepackage{microtype} % Slightly tweak font spacing for aesthetics         
\usepackage[utf8]{inputenc}
\usepackage[T1]{fontenc} % Use 8-bit encoding that has 256 glyphs
%%
%% ---------------- Estilos para el formato de página
%%
\usepackage[marginparwidth=3cm, marginparsep=0.3cm]{geometry}
        \geometry{%
                  tmargin=1.5in,
                  bmargin=1.5in,
                  lmargin=2in,
                  rmargin=1in}
        \reversemarginpar
%%\usepackage{float}
\usepackage{chngpage}
\usepackage{fancyhdr}
        \pagestyle{fancy}
                \fancyhf{}
                \rhead{\sc{Est. Aplicada al Diseño de Exp.}}
                \lhead{\sc\nouppercase{\leftmark}}
                \cfoot{\thepage}
                \rfoot{{\em luciano.aguilera@itcm.edu.mx\/}}
%%
%% ----------------- Figuras
%%
\usepackage[pdftex]{graphicx}
        \graphicspath{ {CAPITULOS/IMAGENES-BIOENERGIA/} }
\usepackage{eso-pic} % Se requiere para poner imágenes de fondo
\usepackage[margin=0.5in, hang, bf, small]{caption}
\usepackage{subcaption}
\usepackage{xcolor}
\usepackage{colortbl}
\definecolor{LightCyan}{rgb}{0.88,1,1}
%\usepackage{color}
\usepackage{wrapfig}
\usepackage[section]{placeins}
\usepackage{tikz}
\usepackage{rotating}
\usepackage{pdflscape}
\usepackage{afterpage}
\setlength{\fboxsep}{10pt}
\usepackage{picinpar}
%\usepackage{palatino,avant}


%%
%% ---------------- Estilos para el referenciado
\usepackage{cite}
\usepackage[sort]{natbib}
\usepackage{apalike}
\usepackage{bibentry}
\usepackage{url}
\usepackage{marginnote}
\renewcommand*{\marginfont}{\color{red}\sffamily\footnotesize}

%% ---------------- Estilos de referenciado utilizando Biblatex
%\usepackage[backend=bibtex, natbib=true, style=authoryear]{biblatex}
%\usepackage[backend=bibtex, natbib=true, style=apa]{biblatex}
%\usepackage[backend=biber, natbib=true, sortlocale=auto, style=apa]{biblatex}
%\usepackage{biblatex}
%     \bibliography{/home/luciano/Documentos/BIBLIOTECA/cursos.bib}
%     \addbibresource{cursos.bib}
%
%% ----------------- Problemas y ejercicios
% \usepackage[thmmarks,standard,thref]{ntheorem}
%     \theoremseparator{.}
%     \theorembodyfont{\upshape}
%     \newtheorem{prob}{Problema}[chapter]

\usepackage{answers}

%% ----------------- Estilos Matemáticos y simbolos
\usepackage{amssymb}
\usepackage{amsthm}
     \theoremstyle{definition}
     \newtheorem{ejmp}{Ejemplo}[section]
\usepackage{amsmath}
\usepackage{amsbsy}
\usepackage{amsfonts}
\usepackage{eurosym}
\usepackage[version=4]{mhchem} % Simbología de reacciones químicas
%%
%%
%% ------------------Estilos de tablas
\usepackage{xtab}
%\usepackage{longtable}
%\usepackage{supertabular}
\usepackage{booktabs} % Mejora la apariencia de las líneas en el ambiente "tabular"
\usepackage{multicol}
\usepackage{multirow}
\usepackage{array}
\newcolumntype{L}[1]{>{\raggedright\let\newline\\\arraybackslash\hspace{0pt}}m{#1}}
\newcolumntype{C}[1]{>{\centering\let\newline\\\arraybackslash\hspace{0pt}}m{#1}}
\newcolumntype{R}[1]{>{\raggedleft\let\newline\\\arraybackslash\hspace{0pt}}m{#1}}
%%
%% ------------------Estilos de párrafos
\usepackage{parskip}
\parindent0pt 
\renewcommand{\baselinestretch}{1.3}
% Control de líneas huerfanas y líneas viudas
\clubpenalty=10000
\widowpenalty=10000
%\setlength{\parskip}{\baselinestretch}
\setlength{\parskip}{1.3\baselineskip}
\usepackage{ragged2e}
\usepackage{verbatim}
\usepackage{textcase}
\usepackage{exercise}

%% ----------------- Listas

\usepackage[inline]{enumitem} % Personaliza listas
                \setlist{nolistsep} % Reduce el espaciado en las listas

%%
%% ----------------- Comandos que requiere Fancyheader
%%                   antes de procesar el texto
%%
\renewcommand{\chaptermark}[1]{%
        \markboth{\thechapter.\ #1}{}}
%
\everymath{\displaystyle}
%%
%% Definición de colores y ambientes
\definecolor{color1}{RGB}{44,42,37}       % Borde general
\definecolor{color2}{RGB}{240,240,240}    % Fondo del cuerpo
\definecolor{color3}{RGB}{44,42,37}       % Fondo del Encabezado
\definecolor{color4}{RGB}{240,240,240}    % Borde del Encabezado


%%
%%
\hyphenation{problema presente moléculas especialmente plantas}



\title{Procesos Estocásticos\\
      {\normalsize \textsc{Estadística Aplicada al Diseño de Experimentos}}}
  \author{Dr. Luciano Aguilera Vázquez}

\begin{document}

\maketitle

\section{Introducción}

El uso del término estocástico hace referencia a la evolución de un
fenómeno aleatorio en el tiempo. Este tipo de fenómeno se estudia con
el objetivo de explicar la estructura del fenómeno y proveer una
predicción de su evolución, al menos a corto plazo, de una variable
que observamos a lo largo del tiempo. Muchos fenómenos han sido
estudiados con un enfoque probabilista, tratándolos como procesos
estocásticos, ya que especificamos las variables que son aleatorias
para cada instante del tiempo.

Siguiendo estos pensamientos, es posible traducir un proceso
estocástico como una colección o familia de variables aleatorias,
ordenadas que pueden variar en el tiempo y, que además, por no
reconocerse el origen o la causa de dichos fenómenos, son no
deterministas.  Esto significa que el estado subsecuente del sistema
se determina tanto por las acciones predecibles del proceso, como por
un elemento aleatorio. La mayoría (si es que no todos), los sistemas
de la vida real son estocásticos.

Su comportamiento puede ser medido y aproximado a distribuciones y
probabilidades, pero rara vez pueden ser determinados por un solo
valor. A continuación se indican algunos ejemplos para explicar los
conceptos descritos:

\begin{itemize}
    \item El tiempo que un cajero de un banco requiere para procesar
    el depósito de un cliente depende de varios factores (algunos de
    ellos pueden ser controlados, otros no; algunos son medibles,
    otros no), pero al final, realizando un conjunto de observaciones
    del tiempo de procesamiento de cada deposito del cajero puede
    permitir ajustar los tiempos a una distribución y ``predecir''
    cuál será el tiempo de proceso en un modelo de simulación por
    eventos discretos.
    \item Supóngase que se estudia el número de personas que asistan
    al servicio médico en cierto hospital. En un intervalo de tiempo
    determinado, digamos una hora, se puede estudiar el comportamiento
    de llegas mediante la definición de la variable aleatoria ``Número
    de personas que llegan el consultorio en una hora''. Si en vez de
    una hora se consideran dos horas, es claro que el número de
    llegadas tiene a ser mayor, y por consiguiente, la distribución de
    probabilidad de esta nueva variable aleatoria será distinta a la
    anterior.  Esto lleva a decir que para cada tiempo, se tendrá una
    nueva variable aleatoria, generalmente distinta. Y por lo tanto,
    una forma muy natural de controlar el estudio es definiendo una
    familia de variables aleatorias, las cuales dependen de una
    variable determinista que es el tiempo. En situaciones como esta,
    decimos que se está trabajando con un proceso estocástico.
    \item Señales de telecomunicación
    \item Señales biomédicas
    \item El número de manchas solares año tras año
\end{itemize}

\section{Definición de procesos estocásticos}

Al seguir una o varias variables en un sistema real cualquiera,
definido por sus fronteras y sus componentes, se obtienen valores
conforme avanza el tiempo ($t$). Cada uno de estos valores nos da
información del fenómeno que se observa y lo hace único para las
condiciones en las que se desarrolla. Estos valores pueden definirse
matemáticamente como $X(t)$, $Y(t)$ o $Z(t)$. A cada conjunto de
valores de estas variables por cada $t$ se le conoce como
\emph{estado}.

Así, ${X_n, n \; \in \; N}$, donde $n$ indica el espacio de
tiempo o instante correspondiente.


\end{document}

%%% Local Variables:
%%% mode: latex
%%% TeX-master: t
%%% End:
